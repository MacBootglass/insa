\documentclass[a4paper,12pt]{article}
\usepackage[french]{babel}
\usepackage[T1]{fontenc}
\usepackage[utf8]{inputenc}
\usepackage{graphicx}
\usepackage{pdfpages}
\usepackage{fancyhdr}

\pagestyle{fancy}
\fancyhead[L]{Thibault THEOLOGIEN}
\fancyhead[R]{Automatique - INSA Rouen - ASI 3.2}

\title{Automatique - Compte rendu TP5}
\author{
	Thibault THEOLOGIEN\\
	INSA Rouen\\
	ASI 3.2 - Groupe 1.1
}

\begin{document}
	\maketitle
	\tableofcontents
	\newpage

  \par L'objectif du mini-projet est la réalisation de la commande par retour d'un pont roulant.

  \section{Détermination du modèle linéaire du pont roulant}
  \label{sec:Détermination du modèle linéaire du pont roulant}
    \par Se référer à l'annexe \ref{sec:Obtention du modèle d'état linéarisé du système} page \pageref{sec:Obtention du modèle d'état linéarisé du système}
    afin de visualiser le calcul de l'obtention du modèle d'état linéarisé du système.

    \begin{figure}[h]
      \caption{Simulation du comportement du chariot}
      \centering
      \includegraphics[width=12cm]{/home/ttheologien/MEGAsync/ASI_3.2/Automatique/TP5/Images/Figure1.png}
    \end{figure}

    \par La vitesse du chariot tend vers l'infini donc le système n'est pas stable.
    Cela se voit aussi en calculant les valeurs propres de la matrice A puisqu'elles sont nulles.
    On se situe donc à la limite de stabilité.
    D'autre part, la position angulaire et la vitesse angulaire oscillent en continu.
  \newpage

  \section{Analyse du modèle linéaire}
  \label{sec:Analyse du modèle linéaire}
    \par Nous trouvons grâce à Scilab que le rang C(A,B) est 4 et nous savons que n = 4 donc le système estcommandable.\\

    \par Cas de la sortie de la position du chariot seule mesurée:
    Le rang de O(A,C) est 4 et n=4 donc la sortie est observable pour C1=[1 0 0 0] car nous avons autant d'entrées que de sorties.\\

    \par Cas de la sortie de l'angle du filin seul mesuré:
    Le rang de O(A,C) est 2 et n=4 donc la sortie n'est pas observable pour C2=[0 0 1 0].\\

    \par Ainsi, le seul cas ou le système est à la fois commandable et observable est celui où la sortie mesurée est celle de la position du charriot.
  \newpage

  \section{Asservissement par retour d'état}
  \label{sec:Asservissement par retour d'état}
    \par Se référer à l'annexe \ref{sec:Calcul des paramètres du système asservi} page \pageref{sec:Calcul des paramètres du système asservi}
    afin de visualiser le calcul des paramètres du système linéarisé.

    \begin{figure}[h]
      \caption{Simulation du comportement du système asservi pour une consigne r(t) = 50m}
      \centering
      \includegraphics[width=12cm]{/home/ttheologien/MEGAsync/ASI_3.2/Automatique/TP5/Images/Figure2.png}
    \end{figure}

    \par Nous remarquons que la position de la charge a varié de 50m, et la vitesse du chariot a augmenté audébut du déplacement pour ensuite diminuer et finalement s’annuler.
    Les variations de la position angulaire et de la vitesse angulaire sont très faibles.
    On peut donc dire que la charge n’a pas trop oscillé pendant le déplacement.
    Notre système a donc été stabilisé.
    On obtient les performances suivantes:
    \begin{itemize}
      \item Le pourcentage de dépassement est de: 4.06\%
      \item Le temps de montée (passage de 10\% à 90\% de la valeur finale) est de: 17.9s
      \item Le temps de réponse à 5\% obtenu est de: 27.2s
    \end{itemize}
    Tous les critères sont remplis, que ce soit au niveau du dépassement ou du temps de montée (le temps de montée obtenu est légèrement inférieur à celui escompté, cependant la marge d'erreur est raisonnable).
    De plus, en régime permanent, l'erreur statique est bien nulle.
  \newpage

  \section{Réalisation d'un observateur}
  \label{sec:Réalisation d'un observateur}
    \par Dans la partie \ref{sec:Analyse du modèle linéaire}, nous avons trouvé que le système était observable si la seule sortie mesurée est la position du chariot.
    Il est donc possible de reconstruire les autres états à partir de la connaissance de xc = x1 et de l'entrée.\\

    \par Nous avons une valeur propre unique $\lambda$ = -2
     Or, nous avons besoin de 4 valeurs propres.
     L’ordre de multiplicité de $\lambda$ est donc 4.
     Ce pôle est plus rapide que les pôles dominants puisqu'il est plus loin de l'axe des imaginaires donc la convergence de l'état estimé grâce à l'observateur vers l'état réel du système est beaucoup plus rapide.\\

    \par Avec l’aide de Scilab, on trouve la matrice de gain de l'observateur:\\
     L = [8.00 19.05 0.00 -1.98]
  \newpage

  \section{Commande par retour d'état avec reconstruction de l'état}
  \label{sec:Commande par retour d'état avec reconstruction de l'état}
    Soit le modèle suivant obtenu: \\
    \[
      \left[
            \begin{array}{c}
              \dot{x_1}\\
              \dot{x_2}
            \end{array}
      \right]
      =
      \left[
            \begin{array}{cc}
              A  & -BK\\
              LC & A-BK-LC
            \end{array}
      \right]
      \times
      \left[
            \begin{array}{c}
              x_1\\
              x_2
            \end{array}
      \right]
      +
      \left[
            \begin{array}{c}
              BS\\
              BS
            \end{array}
      \right]
      \times r
    \]

    \begin{figure}[h]
      \caption{Simulation du comportement du système global en boucle fermée}
      \centering
      \includegraphics[width=12cm]{/home/ttheologien/MEGAsync/ASI_3.2/Automatique/TP5/Images/Figure3.png}
    \end{figure}

    On obtient les performances suivantes:
    \begin{itemize}
      \item Le pourcentage de dépassement est de: 4.06\%
      \item Le temps de montée (passage de 10\% à 90\% de la valeur finale) est de: 17.9s
      \item Le temps de réponse à 5\% obtenu est de: 47.5s
    \end{itemize}

    \par Nous sommes toujours conformes au cahier des charges.
    En effet, à l'exception du temps de réponse à 5\%, non requis au cahier des charges, les performances n'ont pas changé depuis la partie \ref{sec:Asservissement par retour d'état}.
    Il en va de même avec les courbes obtenues dont l'allure est identique.
  \newpage


	\appendix
		\section{Code Scilab}
    \label{sec:Code Scilab}
		\begin{center}
			\includepdf[page=1]{/home/ttheologien/MEGAsync/ASI_3.2/Automatique/TP5/Code.pdf}
			\includepdf[page=2]{/home/ttheologien/MEGAsync/ASI_3.2/Automatique/TP5/Code.pdf}
			\includepdf[page=3]{/home/ttheologien/MEGAsync/ASI_3.2/Automatique/TP5/Code.pdf}
		\end{center}

		\section{Obtention du modèle d'état linéarisé du système}
		\label{sec:Obtention du modèle d'état linéarisé du système}
		\newpage

    \section{Calcul des paramètres du système asservi}
    \label{sec:Calcul des paramètres du système asservi}
    \newpage

\end{document}
