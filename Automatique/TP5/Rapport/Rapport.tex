\documentclass[a4paper,12pt]{article}
\usepackage[french]{babel}
\usepackage[T1]{fontenc}
\usepackage[utf8]{inputenc}
\usepackage{graphicx}
\usepackage{pdfpages}
\usepackage{fancyhdr}

\pagestyle{fancy}
\fancyhead[L]{Thibault THEOLOGIEN}
\fancyhead[R]{Automatique - INSA Rouen - ASI 3.2}

\title{Automatique - Compte rendu TP5}
\author{
	Thibault THEOLOGIEN\\
	INSA Rouen\\
	ASI 3.2 - Groupe 1.1
}

\begin{document}
	\maketitle
	\tableofcontents
	\newpage

  \par L'objectif du mini-projet est la réalisation de la commande par retour d'un pont roulant.

  \section{Détermination du modèle linéaire du pont roulant}
  \label{sec:Détermination du modèle linéaire du pont roulant}
    \par Se référer à l'annexe \ref{sec:Obtention du modèle d'état linéarisé du système} page \pageref{sec:Obtention du modèle d'état linéarisé du système}
    afin de visualiser le calcul de l'obtention du modèle d'état linéarisé du système.

    \begin{figure}[h]
      \caption{Simulation du comportement du chariot}
      \centering
      \includegraphics[width=12cm]{/home/ttheologien/MEGAsync/ASI_3.2/Automatique/TP5/Figure1.png}
    \end{figure}

    \par La vitesse du chariot tend vers l'infini donc le système n'est pas stable.
    Cela se voit aussi en calculant les valeurs propres de la matrice A puisqu'elles sont nulles.
    On se situe donc à la limite de stabilité.
    D'autre part, la position angulaire et la vitesse angulaire oscillent en continu.
  \newpage

  \section{Analyse du modèe linéaire}
  \label{sec:Analyse du modèe linéaire}
    \par Nous trouvons grâce à Scilab que le rang C(A,B) est 4 et nous savons que n = 4 donc le système estcommandable.\\

    \par Cas de la sortie de la position du chariot seule mesurée:
    Le rang de O(A,C) est 4 et n=4 donc la sortie est observable pour C1=[1 0 0 0] car nous avons autant d'entrées que de sorties.\\

    \par Cas de la sortie de l'angle du filin seul mesuré:
    Le rang de O(A,C) est 2 et n=4 donc la sortie n'est pas observable pour C2=[0 0 1 0].\\

    \par Ainsi, le seul cas ou le système est à la fois commandable et observable est celui où la sortie mesurée est celle de la position du charriot.
  \newpage

  \section{Analyse par retour d'état}
  \label{sec:Analyse par retour d'état}
    \par Se référer à l'annexe \ref{sec:Calcul des paramètres du système asservi} page \pageref{sec:Calcul des paramètres du système asservi}
    afin de visualiser le calcul des paramètres du système linéarisé.

    \begin{figure}[h]
      \caption{Simulation du comportement du système asservi pour une consigne r(t) = 50m}
      \centering
      \includegraphics[width=12cm]{/home/ttheologien/MEGAsync/ASI_3.2/Automatique/TP5/Figure2.png}
    \end{figure}
  \newpage

  \section{Réalisation d'un observateur}
  \label{sec:Réalisation d'un observateur}
  \newpage

	\appendix
	% 	\section{Code Scilab}
	% 	\begin{center}
	% 		\includepdf[page=1]{/home/ttheologien/MEGAsync/ASI_3.2/Automatique/TP5/Code.pdf}
	% 		\includepdf[page=2]{/home/ttheologien/MEGAsync/ASI_3.2/Automatique/TP5/Code.pdf}
	% 		\includepdf[page=3]{/home/ttheologien/MEGAsync/ASI_3.2/Automatique/TP5/Code.pdf}
	% 	\end{center}
	%
		\section{Obtention du modèle d'état linéarisé du système}
		\label{sec:Obtention du modèle d'état linéarisé du système}
		\newpage

    \section{Calcul des paramètres du système asservi}
    \label{sec:Calcul des paramètres du système asservi}
    \newpage

\end{document}
