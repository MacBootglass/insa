\documentclass[a4paper,12pt]{article}
\usepackage[french]{babel}
\usepackage[utf8]{inputenc}
\usepackage{graphicx}
\usepackage{pdfpages}

\title{Automatique - Compte rendu TP3}
\author{Thibault THEOLOGIEN}

\begin{document}
	\maketitle
	\tableofcontents
	\newpage
	
	\section{Étude du système mécanique}
		\subsection{Étude théorique}	
			\newpage

		\subsection{Étude pratique}
			\begin{center}
				\includegraphics[width=12cm]{../Images/Figure1.png} 	
				\begin{itemize}
					\item Le temps de réponse à 5 \% obtenu est de :
					\item Le pourcentage de dépassement est de:
					\item Le temps de montée est de: 			
				\end{itemize}
			\end{center}			
	\newpage
			
	\section{Cahier des charges}
		\newpage	
		
	\section{Réglage de la rapidité}
		\subsection{Étude théorique}
			\newpage

		\subsection{Étude pratique: réponse de l'enregistreur asservi}
			\begin{center}
				\includegraphics[width=12cm]{../Images/Figure2.png} 	
				\begin{itemize}
					\item Le temps de réponse à 5 \% obtenu est de :
					\item Le pourcentage de dépassement est de:
					\item Le temps de montée est de: 			
				\end{itemize}
				Au vu des résultats obtenus nous pouvons en déduire deux choses: soit les calculs effectués sont erronés, soit les performances retrouvées ne répondent pas aux exigences du cahier des charges.
				\includegraphics[width=12cm]{../Images/Figure3.png}
				\begin{itemize}
					\item La marge de phase obtenue est de:
					\item Le pulsation de coupure w0 obtenue est de:
				\end{itemize}
			\end{center}


	\section{Réglage du dépassement}
		\subsection{Étude théorique}
			\newpage
		\subsection{Étude pratique} 

	\section{Code Scilab}
		\begin{center}
			\includepdf[page=1]{../Images/Code.pdf}
			\includepdf[page=2]{../Images/Code.pdf}
			\includepdf[page=3]{../Images/Code.pdf}
		\end{center}
\end{document}

