\documentclass[a4paper,12pt]{article}
\usepackage[french]{babel}
\usepackage[T1]{fontenc}
\usepackage[utf8]{inputenc}
\usepackage{graphicx}
\usepackage{pdfpages}

\title{Automatique - Compte rendu TP3}
\author{Thibault THEOLOGIEN}

\begin{document}
	\maketitle
	\tableofcontents
	\newpage

	\section{Etude de la boucle secondaire}
		\subsection{Etude des caractéristiques de H2(p)}
			\begin{center}
				\includegraphics[width=12cm]{/home/ttheologien/MEGAsync/ASI_3.2/Automatique/TP4/Images/Figure1.png}
			\end{center}

			\begin{itemize}
				\item Le temps de réponse à 5 \% obtenu est de: 52.9ms
				\item Le pourcentage de dépassement est de: 689.45\%
				\item Le temps de montée (passage de 10\% à 90\% de la valeur finale) est de: 1.5ms
				\item Le temps de montée (temps pour atteindre pour la première fois la valeur finale) est de: 0.2ms
			\end{itemize}

			Le couple appliqué à la bande est 600 fois trop important.

			\begin{center}
				\includegraphics[width=12cm]{/home/ttheologien/MEGAsync/ASI_3.2/Automatique/TP4/Images/Figure2.png}
			\end{center}

			\begin{itemize}
				\item La marge de phase obtenue est de: 89.68\degre
				\item La marge de gain obtenue est infinie.
				\item La fréquence de coupure à 0 dB fcO obtenue est de: 31830Hz
			\end{itemize}

		\subsection{Synthèse du régulateur de couple}
			Afin que la boucle secondaire puisse satisfaire les exigences du cahier des charges on va utiliser un correcteur PI.\\
			En effet on souhaite avoir une marge de phase de 90\degre, un temps de réponse de 0,4 ms et une erreur statique nulle. \\
			La marge de phase étant déjà correcte, il ne reste donc qu'à régler les deux autres paramètres. \\
			Pour obtenir une erreur statique nulle, il nous faut un intégrateur. Le correcteur PI permet de répondre à ces exigences: il permet de diminuer le temps de réponse et d'annuler l'erreur statique sans changer la marge de phase.
			\newpage
			Calcul des paramètres du correcteur:

			\begin{center}
				\includegraphics[width=12cm]{/home/ttheologien/MEGAsync/ASI_3.2/Automatique/TP4/Images/Figure3.png}
			\end{center}

			On obtient les paramètres suivant:
			\begin{itemize}
				\item La marge de phase obtenue est de: 89.68\degre
				\item La marge de gain obtenue est infinie.
				\item Le pulsation de coupure wc0 obtenue est de: 200000rad/s
				\item La fréquence de coupure à 0 dB fcO obtenue est de: 31830Hz
			\end{itemize}

	\newpage

	\section{Etude de la boucle principale}
		\subsection{Etude des caractéristiques de la boucle principale}

		\subsection{Synthèse du régulateur de traction}

		\subsection{Sensibilité de la régulation}
	\newpage
	\section{Code Scilab}


\end{document}
