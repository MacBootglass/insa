\documentclass[a4paper,12pt]{article}
\usepackage[french]{babel}
\usepackage[utf8]{inputenc}
\usepackage{graphicx}
\usepackage{mathtools}

\DeclarePairedDelimiter\abs{\lvert}{\rvert}

\title{Automatique - Devoir maison numéro 2}
\author{Thibault THEOLOGIEN}

\begin{document}
	\maketitle
	
	\section{TD 5 - Exercice 4}
		\begin{enumerate}
			\item Afin de calculer l'erreur statique $\xi_{c,p}$ il faut déterminer la valeur de $\alpha_0$ telle que $H_{BO}(s)$ = $\frac{K}{s^{\alpha_O}} \times \frac{N(s)}{D(s)}$ soit $\frac{K}{s} \times \frac{1}{(s+1)^2}$ soit $\alpha_0$ = 1. On en déduit que $\xi_{c,p}$ = $\lim\limits_{x \to 0} \frac{s^1}{s^1 + K}$ = 0.
			\item Déterminer la valeur du gain K qui permet d'obtenir une pulsation de coupure à 0dB égale à $\omega_{c0}$ = 2rad/s revient à résoudre le système suivant:
				\[
				\abs{H_{BO}(j\omega_{c0})} = 1
				\]
				\[
					\frac{\abs{K}}
							 {\abs{ j\omega_{c0} (j\omega_{c0} + 1)^2}} = 1
				\] 
				\[
					K = \abs{j\omega_{c0}} \times \abs{j\omega_{c0} +1} \times \abs{j\omega_{c0} +1}
				\]
				\[
					K = \sqrt{\omega_{c0}^2} \times \sqrt{\omega_{c0}^2 + 1} \times \sqrt{\omega_{c0}^2 + 1}
				\]
				\[
					K = \omega_{c0} \times (\omega_{c0}^2 + 1)
				\]
				Étant donné que $\omega_{c0}$ = 2
				\[
					K = 2 \times (2^2 + 1) = 10
				\]
			\item Afin de calculer la marge de phase pour laquelle K=10, ce qui revient au fait que $\omega_{c0}$ = 2, il faut déterminer la valeur de arg($H_{BO}($j$\omega_{c0}$)).
			Soit: 
			\[
				arg(H_{BO}(j\omega_{c0})) = arg(K) - (arg(j\omega_{c0}) + arg((\omega_{c0} +1)^2)
			\]
			\[
				arg(H_{BO}(j\omega_{c0})) =  - \frac{\pi}{2} - Arctan(\frac{\omega_{c0}^2}{1^2})
			\]
			\[
				arg(H_{BO}(j\omega_{c0})) = - \frac{\pi}{2} - 2Arctan(2)
			\]
			La marge de phase $\varphi$ est égale à arg($H_{BO}($j$\omega_{c0}$)) + $\pi$, soit
			\[
				\varphi = - \frac{\pi}{2} - 2Arctan(2) + \pi
			\]
			\[
				\varphi = \frac{\pi}{2} - 2Arctan(2) = -0.643 rad 
			\]
			\[
				\varphi = \frac{-0.643 \times  \pi}{\pi} deg = -36.87 deg
			\]
			La marge de phase $\varphi$ étant négative, on en déduit que le système est instable en BF. 
		\end{enumerate}
\end{document}

