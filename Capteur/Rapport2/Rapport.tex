\documentclass[a4paper,12pt]{article}
\usepackage[french]{babel}
\usepackage[T1]{fontenc}
\usepackage[utf8]{inputenc}
\usepackage{graphicx}
\usepackage{fancyhdr}
\pagestyle{fancy}
\fancyhead[L]{TP Capteur: Haut Parleur}
\fancyhead[R]{ASI 3.2 - INSA Rouen}
\fancyfoot[L]{TORION - THEOLOGIEN}
\fancyfoot[R]{\thepage}
\fancyfoot[C]{}

\title{Capteur - Compte rendu TP haut parleur}
\author{
	Nicolas TORION - Thibault THEOLOGIEN\\
	INSA Rouen\\
	ASI 3.2 - Groupe 1.1
}

\begin{document}
	\maketitle
	\tableofcontents
	\newpage

	\section{Etude de l'accéléromètre en boucle ouverte}
	\label{sec:Etude de l'accéléromètre en boucle ouverte}
		\begin{figure}[!h]
			\caption{Schéma du montage}
			\centering
			\includegraphics[width=12cm]{/home/ttheologien/MEGAsync/ASI_3.2/Capteur/TP5/schemaBO.png}
		\end{figure}

		\par On règle le signal de sortie du GBF à une fréquence de 100Hz et une tension de sortie de 500mV crête-crête qui restera constante.
		On ajuste ensuite le gain de l’amplificateur accéléromètre de manière à obtenir un signal sinusoïdal de même valeur crête que le signal GBF (VACC).

		\begin{figure}[!h]
			\caption{Données obtenues}
			\centering
			\includegraphics[width=12cm]{/home/ttheologien/MEGAsync/ASI_3.2/Capteur/TP5/releveBO.png}
		\end{figure}

		\begin{figure}[!h]
			\caption{}
			\centering
			\includegraphics[width=12cm]{/home/ttheologien/MEGAsync/ASI_3.2/Capteur/TP5/diagrammeGainBO.png}
		\end{figure}

		\begin{figure}[!h]
			\caption{}
			\centering
			\includegraphics[width=12cm]{/home/ttheologien/MEGAsync/ASI_3.2/Capteur/TP5/diagrammePhaseBO.png}
		\end{figure}

	\pagebreak

	\section{Etude de l'accéléromètre en boucle asservie}
	\label{sec:Etude de l'accéléromètre en boucle asservie}
		\begin{figure}[!h]
			\caption{Schéma du montage}
			\centering
			\includegraphics[width=12cm]{/home/ttheologien/MEGAsync/ASI_3.2/Capteur/TP5/schemaBF.png}
		\end{figure}

		\par On positionne le gain de l’amplificateur HP à 0.
		On régle le GBF à 500mV crête-crête.
		On augmente ensuite le gain de l’amplificateur jusqu’à ce qu’il y ait une oscillation puis le diminuer juste en dessous de l’oscillation.

		\begin{figure}[!h]
			\caption{Données obtenues}
			\centering
			\includegraphics[width=12cm]{/home/ttheologien/MEGAsync/ASI_3.2/Capteur/TP5/releveBF.png}
		\end{figure}

		\begin{figure}[!h]
			\caption{}
			\centering
			\includegraphics[width=12cm]{/home/ttheologien/MEGAsync/ASI_3.2/Capteur/TP5/diagrammeGainBF.png}
		\end{figure}

		\begin{figure}[!h]
			\caption{}
			\centering
			\includegraphics[width=12cm]{/home/ttheologien/MEGAsync/ASI_3.2/Capteur/TP5/diagrammePhaseBF.png}
		\end{figure}


\end{document}
